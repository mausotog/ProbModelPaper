
\section{Introduction} \label{introduction}

% no \IEEEPARstart
Repairing errors is one of the most 
resource intensive tasks in 
the software development process~\cite{Weiss07,Tassey02,Britton13}, especially for large, real-world systems~\cite{Liblit03,Anvik05}.
%
Significant recent research effort has been dedicated to
building automatic program repair (APR) tools that are able to address
bugs in 
programs (e.g.~\cite{legoues12,kim2013,Weimer13,long15SPR,long16proph,debroy10,perkins09,wei10}).

One well-known class of repair techniques follows a 
\emph{generate-and-validate} approach  (e.g. GenProg~\cite{legoues12}, 
Par~\cite{kim2013}, TrpAutoRepair~\cite{Qi13TrpAutoR}
Prophet~\cite{long16proph}, HDRepair~\cite{xuan16}, Angelix~\cite{Mechtaev2016},
Nopol~\cite{xuanNopol}).  This approach takes as input a test suite, 
including at
least one failing test case exposing
a defect, and source code to be 
modified.  These approaches then \emph{generate} candidate patches
by applying 
mutation operators to the original source code and \emph{validate} each by
running the potentially patched program against the test suite.  The goal is to
construct a patch that
leads the program to pass all input test cases. 

The search space of possible patches that can repair any bug is trivially
infinite, introducing key tensions in APR
technique design.  First, 
techniques must manage the space carefully so as to traverse it in a scalable and
effective manner.  One way that current techniques manage this problem is to
only consider a limited set of possible changes, or \emph{mutation operators}, 
applied to potentially-fault code (typically identified using off-the-shelf fault 
localization, e.g. Tarantula~\cite{Jones02}).  
Techniques differ in the mutation operators they consider, and how those operators
are instantiated.  Syntactically-informed APR techniques typically instantiate
mutations by selecting repair code heuristically from elsewhere in the same
file, module, or program.  As a simple example, GenProg~\cite{legoues12Genprog} can either
delete program statements, replace one statement for another, or insert one
statement after another; the ``fix'' statements are selected from elsewhere in
the same program.  There is a broad diversity of such syntactic operators used
by other APR techniques, including more complicated templated modifications~\cite{kim2013} and transformation 
schemas~\cite{long16proph,long15SPR}.  Semantically-informed
techniques~\cite{nguyen13,Mechtaev2016,xuanNopol} differ in that they use synthesis to
construct the fix code used in instantiating program modifications.  
Regardless, any such technique uses heuristics or heuristically-informed
probability distributions to select between either mutation operators or fix
code (or both) when constructing candidate patches. 

However, even with these design choices, the search space of candidate
repairs for many bugs remains intractably large.  Most techniques are thus
limited by design to producing single-edit patches (e.g. TrpAutoRepair~\cite{Qi13TrpAutoR}). 
Those few that can theoretically produce multi-edit patches
typically still only produce small changes in
practice~\cite{Qi15,arcuri11,Mechtaev2016}. 
Indeed, patches
consisting of several mutations are undertreated in the APR context, even though the vast majority
of real bug-fixing patches are comprised of multiple edits~\cite{Soto16,zhong15}.

These attempts to tackle scalability are in tension with a second major concern
in APR research: the quality of the generated patches~\cite{Qi15}.  One way to
improve output quality is to use more expressive mutation operators, producing a
richer search space~\cite{long16}.  However, more and more expressive mutation
operators increase the size of the 
space, and not always in a way that supports the construction of more correct
patches.  Thus, although APR techniques would benefit from more expressive
operators, they cannot do so without a way to choose between them that manages
the search space explosion problem, ideally while increasing the probability
of finding truly correct patches.

In bug-fixing reality, human developers use
certain mutation operators much more often than others (e.g. adding an \emph{IfStatement} is more common than adding a \emph{TypeDeclarationStatement})~\cite{Soto16}. In this
work, we propose to study
and then systematically simulate the behavior of human developers to create
patches. This idea has been leveraged manually, to create
mutation operators that produce more human-acceptable patches~\cite{kim2013} and
automatically, to inform patch \emph{ranking} (rather than
construction) when traversing the search space~\cite{xuan16,long16proph}. 
Our key supposition is that our approach can scalably support a richer
search space constructed via more expressive mutation operators that are more
likely to produce high-quality patches.  

To this end, we mine bug fixing commits from 
the 500 most popular Github Java projects to model the selection probability of
the possible mutation operators based on 
empirical data that describes how human programmers fix their code.  
The APR search space, including the various mutation
operators used by different techniques, has been growing in a relatively ad hoc 
manner through the various approaches proposed in recent years. We thus
categorize, compare, and validate a superset of
mutation operators in use in a number of 
state-of-the-art approaches~\cite{legoues12,Weimer13,kim2013,long16proph}.
We use this model to inform a repair approach that chooses from the set of possible
operators based on these real-world probabilities.
As a result, our work goes beyond prior work  by generalizing
to a broader and more expressive set of
mutation operators, and using a fully-automatically-mined model much 
earlier in the APR process, when patch candidates are actually \emph{created.}
Furthermore, we propose and initially validate a new approach for modeling
multi-edit repairs based on mining previous rules from historical edit data, 
predicting from a given 
single-edit mutation which operation should be applied
next.  This serves  as a first step towards
scalable traversal of the multi-edit patch space. 

We evaluate the predictive power of our mined models on their own, in terms of their
accuracy in predicting the operators used in real-world bug fixes.  We also
integrate the single-edit model into a new repair algorithm that we demonstrate
in comparison to several previous techniques on a subset of real-world defects from Defects4j~\cite{just14}. 
%
The primary contributions of this paper thus are:
\begin{itemize}
\item \textbf{Empirical model of single-edit repairs to Java code} mined
      from the 100 most recent bug fixes from the 500 most popular Java projects on
    Github. This study also provides a deeper
    understanding of which mutation operators are used by programmers to fix
    errors in source code, and the frequency of each of the mutation operators
    analyzed. 
	\item \textbf{New Repair Approach}, which uses mutation operators
    from multiple state-of-the-art approaches and
    the above-mentioned models to choose between them, favoring those more commonly
    used by human developers.  We validate the use of this model, as well as a
    good set of operators to use for expressive, high-quality repair.
    \item \textbf{Development and integration} of this approach for the Java
      programming language in a tool, available online, which can apply the
      several different mutation operators, and the probabilistic model
      discussed in this study.\footnote{We will release all artifacts, code, and
        data, including generated test suites, associated with this publication post blind-review.} %\footnote{https://bitbucket.org/clegoues/genprog4java}
  \item \textbf{Comparative evaluation} of the new repair approach on real world bugs from well-known open source projects (independent of those used to inform model
    construction). We independently evaluate patch quality using a held-out test
    suite. We conclude that by creating candidate patches using human behavior
    we are able to find patches faster than a technique that does not use such
    mined probabilities, and that the patch creation
    process benefits from combining mutation operator categories. We compare to
    other state-of-the-art tools on the same bugs, assessing expressiveness and
    output quality. 
  \item \textbf{Multi-edit repair approach} A proposed technique, and an initial
    validation thereof, to construct 
    bug-fixing patches requiring several mutations by mining and then applying association
    rules from a large set of historical bug fixes. 
\end{itemize}

The rest of this paper proceeds as follows: We first outline
generate-and-validate repair and, in particular, 
categorize and generalize the mutation operators used in state of the art
approaches (Section~\ref{background}). Section~\ref{buildingTheModel}
describes our approach to building probabilistic models of human 
edits for both single- and multi-edit bug fixes. Section~\ref{evaluation}
describes our evaluation. Section~\ref{relatedWork} describes
related work; Section~\ref{conclusion} talks about the conclusions found in this study. 

